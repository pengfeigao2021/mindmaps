% generate an example tex article
\documentclass{article}
\usepackage{tikz}
\usepackage{hyperref}
\usetikzlibrary{shapes.geometric, arrows}

\begin{document}

\title{Clustering}
\author{Pengfei Gao}
\date{\today}
\maketitle
\tableofcontents

\section{SKLearn}

%% Output a table of clustering algorithms in SKLearn
\begin{table}[h!]
\centering
\begin{tabular}{|c|c|}
\hline
Algorithm & Description \\
\hline
KMeans & convex shaped \\
 & n-dim distance\\
 & centroids: mean position \\
\hline
DBSCAN & Density-based clustering \\
& eps, min samples \\
& not convex shaped \\
\hline
HDBSCAN & Hierarchical Density-Based Spatial Clustering of Applications with Noise \\
& allow different cluster density \\
& $d_c$ n-th nearest neighbors distance \\
& $d_m(x_p, x_q) = max\{d_c(x_p), d_c(x_q), d(x_p, x_q)\}$ \\
& build a fully connected graph and cut edges by threshold \\
\hline
Spectral Clustering & Graph-based clustering \\
& low dim embedding \\
& A: Adjacency matrix \\
& D: Degree matrix \\
& L: Laplacian matrix \\
& L = D - A\\
& L set all diagnal elements to 0 \\
& choose top k eigenvectors of L \\
& these are the clusters with minimum distance \\
\end{tabular}
\caption{Example clustering algorithms in SKLearn}
\end{table}




\end{document}