% simple document template
\documentclass{paper}
\usepackage{tikz}
\usetikzlibrary{graphs}
\usetikzlibrary {mindmap}
\usetikzlibrary {positioning}
\usetikzlibrary {arrows.meta}
\usetikzlibrary {shapes.multipart}
\usetikzlibrary {shapes.geometric}
\usetikzlibrary {shapes.symbols}
\usetikzlibrary {shapes.callouts}
\usetikzlibrary {calc}
\begin{document}

% #ff9393	(255,147,147)
% #fdbcb4	(253,188,180)
% #ffefd5	(255,239,213)
% #b9d9eb	(185,217,235)
% #85b2b0	(133,178,176)

\section{LightGBM paper}
\subsection{2017 LightGBM paper: A Highly Efficient Gradient Boosting Decision Tree}
\begin{itemize}
\item LightGBM is a gradient boosting framework that uses tree-based learning algorithms.
\item Microsoft
\item Efficient and scalable
\item scan all the data instances to estimate the information gain
\item 2 techniques
\item GOSS and EFB
\item GOSS: Gradient-based One-Side Sampling
\item GOSS: remove small gradients
\item EFB: Exclusive Feature Bundling
\item EFB: remove mutual exclusive features
\item results: x20 speedup
\end{itemize}

\subsection{Paramter tunning}
\begin{itemize}
    \item Parameters Tuning
\item \textbf{num\_boosting\_rounds}: number of boosting iterations
\item \textbf{learning\_rate}: step size shrinkage to prevent overfitting
\item \textbf{num\_leaves}: maximum tree leaves
\item \textbf{feature\_fraction}: fraction of features to be randomly selected at each split
\item \textbf{bagging\_fraction}: fraction of training data to be randomly selected for each tree
\item \textbf{bagging\_freq}: frequency of bagging
\end{itemize}

\subsection{FLAML: Auto Hyperparam tunning}
\begin{itemize}
    \item FLAML can choose between lightgbm and xgboost
\end{itemize}


\end{document}