% simple document template
\documentclass{paper}
\usepackage{tikz}
\usetikzlibrary{graphs}
\usetikzlibrary {mindmap}
\usetikzlibrary {positioning}
\usetikzlibrary {arrows.meta}
\usetikzlibrary {shapes.multipart}
\usetikzlibrary {shapes.geometric}
\usetikzlibrary {shapes.symbols}
\usetikzlibrary {shapes.callouts}
\usetikzlibrary {calc}
\begin{document}
% ==========[Section How to make him comfortable around other babies]===========
\section{How to make him comfortable around other babies}
\subsection{chatgpt}

To help your 1.5-year-old son become more comfortable around other babies, you can take a gradual and supportive approach:

% bold
\textbf{Start with Small, Controlled Playdates}: Arrange short playdates with one or two babies in familiar settings. Keep them brief to avoid overwhelming him.

\textbf{Lead by Example}: Show a calm and friendly attitude when interacting with other babies. Your son may pick up on your behavior and feel more at ease.

\textbf{Introduce Social Situations Gradually}: Take him to places where he can observe other babies from a distance before gradually moving closer. This helps him acclimate without pressure.

\textbf{Praise and Encourage}: When he approaches or interacts with other babies, offer praise and positive reinforcement. Keep the atmosphere relaxed and fun.

\textbf{Bring Familiar Comfort Items}: A favorite toy or blanket can help your son feel more secure in new environments.

\textbf{Model Gentle Play}: Gently guide him in interacting with other babies, like showing how to share toys or play together. This helps him understand what to expect.

\textbf{Observe and Respect Boundaries}: If he shows discomfort, give him time and space without forcing interactions. Pushing too hard may reinforce fear.

With patience, he’ll likely grow more comfortable over time.

\subsection{William Sears B01A887NXA}
\paragraph{How to handle children who likes to grab things}~\\
\begin{itemize}
    \item do not directly remove item from children's hand, this will make them angry
    \item distract their attention with something else
    \item tell them we will help them to get those things and teach them politeness
\end{itemize}

\end{document}